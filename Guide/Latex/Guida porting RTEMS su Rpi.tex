\documentclass[10pt, a4paper]{article}
\usepackage{geometry}
 \geometry{
 a4paper,
 total={170mm,257mm},
 left=20mm,
 top=20mm,
 }
 \usepackage{hyperref}
\hypersetup{
    colorlinks=true,
    linkcolor=blue,
    filecolor=magenta,      
    urlcolor=cyan,
}
\title{Guida porting RTEMS su Raspberry Pi}
\author{Clark Ezpeleta}
\usepackage{parskip}
\usepackage{listings}

\begin{document}
\pagenumbering{gobble}
\maketitle
\newpage
\tableofcontents
\newpage
\pagenumbering{arabic}
\begin{flushleft}
\section{Introduzione}
Guida per l'installazione di RTEMS RSB e della tool-suite per l' utilizzo del kernel di RTEMS per l'architettura ARM con target le BSP di Raspberry Pi 1 e Raspberry Pi 2, quest'ultima è compatibile con la Raspberry pi 3.

Il RSB (RTEMS Source Builder) è un tool per buildare i pacchetti dai file sorgenti. Viene usato in RTEMS per buildare i suoi compilatori e OS.

La BSP (Board Support Package) è il codice di supporto per una specifica scheda, esso contiene librerie di RTEMS utili per l'utilizzo della scheda.

Poichè RTEMS è un progetto open source ancora in sviluppo, al momento stanno sviluppando la v6, ma non è ancora stabile, per questo motivo in questa guida si installerà la v5.1 che è la versione stabile più recente.

Durante questa guida verranno utilizzati comandi da terminale, e il sistema operativo su cui si basa è Ubuntu LTS 20.04.


\newpage
\section{Configurazione iniziale del sistema}
Prima di iniziare l'installazione di RTEMS RSB e tool-suite bisogna configurare l'ambiente di sviluppo.

Procediamo con l'installazione dei seguenti pacchetti:
\begin{itemize}
\item  build-essential : 
\begin{lstlisting}[language=bash] 
$ sudo apt-get install build-essential
\end{lstlisting}
\item git:
\begin{lstlisting}[language=bash] 
$ sudo apt-get install git
\end{lstlisting}
\item python-dev:
\begin{lstlisting}[language=bash] 
 $ sudo apt-get install python-dev
 \end{lstlisting}
\end{itemize}

A questo punto definiamo la struttura delle cartelle in cui verranno installati i componenti di RTEMS.

\begin{itemize}
\item \$HOME/rtems-dev  : base directory
\item \$HOME/rtems-dev : base directory dove vengono clonati da git la tool-suite e la RSB di RTEMS
\item \$HOME/rtems-dev/src/rsb : RTEMS source builder
\item \$HOME/rtems-dev/src/rtems : RTEMS tool-suite
\item \$HOME/rtems-dev/rtems/ : dove verrà installata la RTEMS tool-suite
\item \$HOME/rtems-dev/build : dove verranno installate le BSP di Rpi1 e Rpi2
\end{itemize}
\newpage
\section{Installazione RTEMS}

Dopo aver preparato l'ambiente di sviluppo, possiamo procedere con l'installazione di RTEMS:
\begin{itemize}

\item Clonazione di RTEMS RSB e tool-suite : 
\begin{lstlisting}[language=bash] 
$ mkdir - p $HOME/rtems-dev/src
$ cd $HOME/rtems-dev/src
$ git clone -b 5.1 git://git.rtems.org/rtems-source-builder.git rsb
$ git clone -b 5.1 git://git.rtems.org/rtems.git
\end{lstlisting}

\item Installiamo la tool suite utilizzando la RSB :
\begin{lstlisting}[language=bash] 
$ cd $HOME/rtems-dev/src/rsb/rtems
$ ../source-builder/sb-set-builder --prefix= $HOME/quick-start/rtems/5 5/rtems-arm
\end{lstlisting}

\item Dopo aver completato l'installazione possiamo controllare che il C cross compiler di RTEMS funzioni : 
\begin{lstlisting}[language=bash] 
$ $HOME/rtems-dev/rtems/5/bin/arm-rtems5-gcc --version
\end{lstlisting}	

\item Inseriamo nel variabili di  ambiente i comandi della toolchain e procediamo con il bootstrap : 
\begin{lstlisting}[language=bash] 
$ export PATH=$HOME/rtems-dev/rtems/5/bin:"$PATH"
$ cd $HOME/rtems-dev/src/rtems
$ ./rtems-bootstrap
\end{lstlisting}	
	
\item Adesso possiamo configurare ed installare le BSPs che ci servono  : 
\begin{lstlisting}[language=bash] 
$ mkdir -p $HOME/rtems-dev/build
$ cd $HOME/rtems-dev/build
$ $HOME/rtems-dev/src/rtems/configure \
        --prefix=$HOME/quick-start/rtems/5 \
	--target=arm-rtems5 \
	--enable-rtemsbsp="raspberrypi raspberrypi2"\
	--enable-tests=samples --enable-networking --enable-posix
$ make
$ make install	
\end{lstlisting}
		
\end{itemize}

Svolti tutti i passaggi precedenti abbiamo come risultato RTEMS installato sul computer host.
\newpage
\section{Prova sample test RTEMS}

Installando le BSP, RTEMS ci fornisce dei sample test .exe da cui possiamo generare i file .img e utilizzarli per testare il funzionamento della raspberry pi

I sample test sono in : 
\begin{itemize}
\item  rpi 1:
\begin{lstlisting}[language=bash] 
$ $HOME/rtems-dev/build/arm-rtems5/c/raspberrypi1/testsuites/samples
\end{lstlisting}
\item  rpi 2 e 3:
\begin{lstlisting}[language=bash] 
$ $HOME/rtems-dev/build/arm-rtems5/c/raspberrypi2/testsuites/samples
\end{lstlisting}
\end{itemize}

In questa guida utilizziamo il sample 'ticker.exe', ma i passaggi che verranno illustrati valgono anche per gli altri sample presenti in cartella.
Per poter utilizzare il sample test dobbiamo fare 2 passaggi:
\begin{itemize}
\item Creazione kernel file .img:

assicurarsi di avere come variabile di ambiente i comandi della tool-suite:
\begin{lstlisting}[language=bash] 
$ echo $PATH
\end{lstlisting}
se non è presente  '\$HOME/rtems-dev/rtems/5/bin' allora bisogna inserirla.

posizionarsi nella cartella dove verrà creato il file .img:
\begin{lstlisting}[language=bash] 
$ cd $HOME/rtems-dev/rtems 
\end{lstlisting}		

generare il file .img:

Per Rpi1: 
\begin{lstlisting}[language=bash] 
$ arm-rtems5-objcopy -Obinary $HOME/rtems-dev/build/arm-rtems5/c/raspberrypi1/ 
testsuites/samples/ticker.exe ticker.img
\end{lstlisting}
Per Rpi2 e Rpi3: 
\begin{lstlisting}[language=bash ] 
$ arm-rtems5-objcopy -Obinary $HOME/rtems-dev/build/arm-rtems5/c/raspberrypi2/
testsuites/samples/ticker.exe ticker.img
\end{lstlisting}
\item Configurare la SD card:
copiamo nella scheda sd il firmware di Rpi (v4.19.11.3) compatibile con RTEMS. 
Il firmware puo' essere scaricato da questo link: 
\url{https://github.com/raspberrypi/firmware/tree/5574077183389cd4c65077ba18b59144ed6ccd6d/boot}
Eliminiamo tutti i file kernel*.img, questi verranno sostituiti dal file 
kernel che abbiamo generato precedentemente.
Copiamo nella sd il file ticker.img creato precendetemente.
Creiamo nella sd il file config.txt che contiene il seguente testo:
\begin{lstlisting}[language=bash ] 
		enable_uart=1
		kernel_address=0x200000
 		kernel=ticker.img 
\end{lstlisting}
Nel campo kernel mettiamo il nome del kernel file che vogliamo che rpi esegua.
\end{itemize}
	 	 
A questo punto siamo pronti con il test.

Un comando per vedere la log della UART per debug è :
\begin{lstlisting}[language=bash] 
$ sudo minicom -b 115200 -D /dev/serial/
by-id/<indirizzo periferica utilizzata per UART di solito un USB TTL>
\end{lstlisting}

Colleghiamo il UART del Rpi al computer e vediamo il risultato:

\newpage
\section{Configurazione Eclipse C/C++} 
Per creare i file sorgenti per programmi di RTEMS possiamo utilizzare l'IDE Eclipse C/C++ scaricabile da questo link :
\url{https://www.eclipse.org/downloads/packages/}


Una volta installato Eclipse C/C++ bisogna installare il plugin di RTEMS:
\begin{itemize}
\item Andiamo su Help> Install New Software
\item Aggiungiamo come Software site quello di RTEMS, quindi clicchiamo add e inseriamo il seguente url \url{ftp://ftp.rtems.org/pub/rtems/eclipse/updates}
\item Selezioniamo il plugin RTEMS CDT Support e installiamo
\end{itemize}

A questo punto abbiamo installato il plugin di RTEMS ed è pronto ad essere utilizzato.

Quando creiamo un progetto RTEMS bisogna configurare la base e il BSP path di RTEMS in modo che si riesca ad utilizzare le librerie che ci servono:
\begin{itemize}
\item Window >Preferences > C/C++ > RTEMS
\item Base path : base path dell'installazione , nel nostro caso home/<username>/rtems-dev/rtems/5
\item BSP path : path dell'installazione della BSP, nel nostro caso home/<username>/rtems-dev/rtems/5/arm-rtems5/raspberrypi2 (o 1 se si utilizza Rpi1)
\end{itemize}

Abbiamo completato la configurazione del plugin di RTEMS su Eclipse C, adesso possiamo procedere alla creazione di un progetto RTEMS:
\begin{itemize}
\item New project>C Project
\item Selezionare il project type corretto: Others> RTEMS Executable
\item Selezionare la Toochain corretta : RTEMS Toolchain
\end{itemize}

Possiamo controllare che il progetto è stato creato correttamente controllando le sue properties e su C/C++ Build> RTEMS i path siano quelli corretti.

\newpage
\section{Prova con Eclipse C/C++}

\newpage
\section{Creazione file kernel senza IDE}

\newpage
\section{Troubleshooting}

\newpage
\section{Bibliografia}

\end{flushleft}


\end{document}
