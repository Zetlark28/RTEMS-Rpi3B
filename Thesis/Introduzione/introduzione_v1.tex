\documentclass[a4paper,12pt,titlepage,oneside]{book}
\title{Bozza introduzione}
\author{Clark Ezpeleta}
\begin{document}
\maketitle
\newpage
\chapter*{Introduzione}

%%Lo scopo del mio lavoro consisteva nel porting del sistema operativo RTEMS su Raspberrypi e nella creazione di piccoli programmi per testare il funzionamento dei GPIO, USART, I2C, SPI e l'utilizzo degli interrupts.
%Per poter svolgere il porting ho dato una lettura al RTEMS User Manual[1] per comprendere meglio i concetti di RSB(RTEMS Source Builder) e BSP (board support packages), e ho utilizzato la guida fornita da un utente - AlanC,  che ha già svolto il porting per la versione 4.11.
%Purtroppo la guida fornita non era totalmente corretta, poichè datata a Marzo 2013 ed era stata scritta per la versione di RTEMS precedente a quella che dovevo utilizzare, cioè la versione 5.1. 
%Per questo motivo dopo aver effettuato correttamente il porting ho creato una guida con scritto tutti i passaggi effettuati integrando le correzioni necessarie rilevate. \\
%Oltre alla Raspberrypi 3B+ che mi è stata fornita, per poter creare i programmi di test dell'interfaccia I2C e SPI erano necessari dei componenti aggiuntivi che mi sono stati gentilmente offerti dalla Microchip.\\\\
%Premesso ciò, ho suddiviso la mia tesi nei seguenti capitoli:

%\begin{itemize}
%   \item \textbf{Capitolo 1} - in questo capitolo vengono descritti gli obiettivi della tesi, il contesto di riferimento e l'applicazione .
%   \item \textbf{Capitolo 2} - in questo capitolo vengono descritte le tecnologie utilizzate.
%   \item \textbf{Capitolo 3} - in questo capitolo viene descritta tutta l'attività di porting svolta e la definizione della toolchain per poter realizzare applicazioni RTEMS.
%   \item \textbf{Capitolo 4} - in questo capitolo vengono descritti i drivers di RTEMS e come le ho provate.
%   \item \textbf{Capitolo 5} - in questo capitolo viene descritto ciò che si vuole testare, in che modo si vuole testare e i motivi per cui si vuole testare quella specifica interfaccia. Perciò vengono descritti i componenti aggiuntivi forniti da Microchip, e l'architettura generale dei programmi di prova.
%   \item \textbf{Capitolo 6} - Ringraziamenti.
%   \end{itemize} 
%La tesi sarà utile alla BIS-Italia, fazione italiana della British Interplanetary Society società storica britannica,di cui sono membro, che mi ha seguito durante lo stage e integrerà ciò al progetto di creazione di una replica in scala 1:3 di ExoMars Rover che verrà utilizzato per divulgazione.
%Tutto il lavoro è stato svolto con l'aiuto dei membri della BIS-Italia e la collaborazione con Microchip.

\section*{Obiettivi}
Gli obiettivi del mio lavoro sono sostanzialmente due :
\begin{itemize}
\item effettuare il porting del sistema operativo RTEMS v5.1 su Raspberrypi 3B+ creando anche una eventuale guida
\item realizzare degli applicativi RTEMS per testare le API delle interfacce GPIO, I2C, SPI e gli interrupts
\end{itemize}
Per il primo obiettivo ho dovuto leggere l'RTEMS User Manual[1] per poter comprendere in linea generale il suo funzionamento.\\
Oltre al User Manual ho potuto utilizzare una guida sul porting della versione 4.1 [2],  creata da un utente - AlanC, per  poter iniziare ad effettuare il porting sulla Raspberrypi 3B+ che mi è stata fornita da BIS-Italia.\\
Dopo aver effettuato il porting con successo, poiché la documentazione e la guida (scritta per una versione differente) sono state scritte in inglese ho creato una guida in italiano che raggruppa tutti i passaggi integrando delle correzioni necessarie che sono state individuate durante il porting.
Per il secondo obiettivo ho dovuto leggere l'RTEMS Classic API Guide [4] e i codici sorgente di esempio trovati in due git repository, per poter comprendere l'utilizzo delle API di RTEMS.

\section*{Contesto di riferimento}
Durante le missioni spaziali bisogna essere sicuri che lo strumento che abbiamo mandato in orbita, riesca ad reagire in modo tempestivo agli imprevisti.
RTEMS che sta per Real-Time Executive for Multiprocessor System, è un Sistema Operativo Real Time open source validato dall'ESA (European Space Agency), che permette di soddifare questo ul requisito sopraesposto.
\section*{Applicazione}
Il lavoro svolto verrà integrato nel progetto del'associazione BIS-Italia (fazione italiana della British Interplanetary Society), di cui sono membro,  che consiste nella creazione di una replica in scala 1:3 di ExoMars Rover dove verrà utilizzato come computer di bordo un Raspberrypi 3B+ su cui girerà un applicativo RTEMS.\\
Per poter simulare la comunicazione tra ExoMars e la stazione di terra, abbiamo bisogno di utilizzare le interfacce GPIO,I2C,SPI e il CAN Bus. Quest'ultima su Rpi non è disponibile e quindi bisognerebbe creare un modulo esterno, ma ciò non verrà trattato nel mio lavoro.\\
Gli applicativi RTEMS di test creati, servono per verificare il corretto funzionamento di queste interfacce della scheda Rpi prima di integrarlo alla struttura HW del rover.


\end{document}