\documentclass[a4paper,12pt,titlepage,oneside]{book}
\linespread{1.2}
\pagestyle{plain}

\usepackage[italian]{babel} 
\usepackage{picture}
\usepackage{hyperref}
\usepackage[Symbol]{upgreek}
\usepackage{amsmath}
\usepackage{listings}
%\usepackage{showframe}
\usepackage{epigraph}
\usepackage{listings}
\usepackage{float}
\usepackage{algorithm, algpseudocode}
\usepackage{tikz}
\usepackage{geometry,lipsum,graphicx}
\lstset{breaklines=true}

\usepackage{pdfpages}
\textheight24cm\topmargin0mm\headheight0mm\headsep6mm\oddsidemargin20pt\evensidemargin30pt

\usepackage[T1]{fontenc}
\usepackage{titlesec, blindtext, color}
\definecolor{gray75}{gray}{0.75}
\newcommand{\hsp}{\hspace{20pt}}
\titleformat{\chapter}[hang]{\Huge\bfseries}{\thechapter\hsp\textcolor{gray75}{|}\hsp}{0pt}{\Huge\bfseries}

\usepackage{csquotes}
\usepackage[backend=bibtex]{biblatex}
\bibliography{bibliografia}
\usepackage{enumitem}




\usepackage{listings}
\usepackage{xcolor}
\usepackage{subcaption}

\definecolor{cyan}{rgb}{0.0,0.6,0.6}

\definecolor{dkgreen}{rgb}{0,0.6,0}
\definecolor{dred}{rgb}{0.545,0,0}
\definecolor{dblue}{rgb}{0,0,0.545}
\definecolor{lgrey}{rgb}{0.9,0.9,0.9}
\definecolor{gray}{rgb}{0.4,0.4,0.4}
\definecolor{darkblue}{rgb}{0.0,0.0,0.6}
\lstdefinelanguage{cpp}{
      %backgroundcolor=\color{lgrey},  
      basicstyle=\scriptsize \ttfamily \color{black} \bfseries,   
      aboveskip=5mm,
      belowskip=-5mm, 
      breakatwhitespace=false,       
      breaklines=true,               
      captionpos=b,                   
      commentstyle=\color{dkgreen},   
      deletekeywords={...},          
      escapeinside={\%*}{*)},                  
      frame=bt,                  
      language=C++,                
      keywordstyle=\color{purple},  
      morekeywords={BRIEFDescriptorConfig,string,TiXmlNode,DetectorDescriptorConfigContainer,istringstream,cerr,exit}, 
      identifierstyle=\color{black},
      stringstyle=\color{blue},      
      numbers=none,                 
      numbersep=5pt,                  
      numberstyle=\tiny\color{black}, 
      rulecolor=\color{black},        
      showspaces=false,               
      showstringspaces=false,        
      showtabs=false,                
      stepnumber=1,                   
      tabsize=5,                     
      title=\lstname,                 
    }
    
    \lstdefinelanguage{xm}{
      %backgroundcolor=\color{lgrey},  
      basicstyle=\scriptsize,
      aboveskip=5mm,
      belowskip=-5mm, 
      belowcaptionskip=0mm,  
      breakatwhitespace=false,       
      breaklines=true,               
      captionpos=b,                   
      commentstyle=\color{gray},            
      frame=bt,       
        morestring=[b]",
  morestring=[s]{>}{<},
  morecomment=[s]{<?}{?>},
  stringstyle=\color{black},
  identifierstyle=\color{darkblue},
  keywordstyle=\color{cyan},
  morekeywords={xmlns,version,type,ira_open_street_map}           
      %language=XML,      
      numbers=none,                 
      numbersep=5pt,                  
      numberstyle=\tiny\color{black},       
      showspaces=false,               
      showstringspaces=false,        
      showtabs=false,                
      stepnumber=1,                   
      tabsize=5,
      escapeinside={<@}{@>},                     
      title=\lstname,                 
    }
    
\iffalse
\lstset{ %
  %language=XML,                % the language of the code
  basicstyle=\tiny,           % the size of the fonts that are used for the code
  numbers=left,                   % where to put the line-numbers
  numberstyle=\tiny\color{gray},  % the style that is used for the line-numbers
  stepnumber=1,                   % the step between two line-numbers. If it's 1, each line  will be numbered
  numbersep=5pt,                  % how far the line-numbers are from the code
  backgroundcolor=\color{white},      % choose the background color. You must add \usepackage{color}
  showspaces=false,               % show spaces adding particular underscores
  showstringspaces=false,         % underline spaces within strings
  showtabs=false,                 % show tabs within strings adding particular underscores
  frame=none,                   % adds a frame around the code
  rulecolor=\color{black},        % if not set, the frame-color may be changed on line-breaks within not-black text (e.g. commens (green here))
  tabsize=1,                      % sets default tabsize to 2 spaces
  captionpos=b,                   % sets the caption-position to bottom
  breaklines=true,                % sets automatic line breaking
  breakatwhitespace=false,        % sets if automatic breaks should only happen at whitespace
%   title=\lstname,                   % show the filename of files included with \lstinputlisting;
                                  % also try caption instead of title
  %identifierstyle=\color{magenta},
  keywordstyle=\color{blue},          % keyword style
  commentstyle=\color{dkgreen},       % comment style
  stringstyle=\color{mauve},         % string literal style
  escapeinside={-*}{*-},            % if you want to add a comment within your code
  morekeywords={*,...}               % if you want to add more keywords to the set
}
\fi

\title{Bozza Introduzione}
\author{Clark Ezpeleta}


\begin{document}

\usepackage[italian]{babel} 
\usepackage{picture}
\begin{document}
\pagenumbering{gobble}
\maketitle
\newpage
\chapter*{Prefazione}

Lo scopo del mio lavoro consisteva nel porting del sistema operativo RTEMS su Raspberrypi e nella creazione di piccoli programmi per testare il funzionamento dei GPIO, USART, I2C, SPI e l'utilizzo degli interrupts.
Per poter svolgere il porting ho dato una lettura al RTEMS User Manual[1] per comprendere meglio i concetti di RSB(RTEMS Source Builder) e BSP (board support packages), e ho utilizzato la guida fornita da un utente - AlanC,  che ha già svolto il porting per la versione 4.11.
Purtroppo la guida fornita non era totalmente corretta, poichè datata a Marzo 2013 ed era stata scritta per la versione di RTEMS precedente a quella che dovevo utilizzare, cioè la v5.1. 
Per questo motivo dopo aver effettuato correttamente il porting ho creato una guida con scritto tutti i passaggi effettuati integrando le correzioni necessarie rilevate. \\
Oltre alla Raspberrypi 3B+ che mi è stata fornita, per poter creare i programmi di test per l'interfaccia I2C e SPI erano necessari dei componenti aggiuntivi che mi sono stati gentilmente offerti dalla Microchip.\\\\
Premesso ciò, ho suddiviso la mia tesi nei seguenti capitoli:

\begin{itemize}
    \item \textbf{Capitolo 1} - in questo capitolo vengono descritti gli obiettivi della tesi, il contesto di riferimento e l'applicazione .
    \item \textbf{Capitolo 2} - in questo capitolo vengono descritte le tecnologie utilizzate.
    \item \textbf{Capitolo 3} - in questo capitolo viene descritta tutta l'attività di porting svolta e la definizione della toolchain per poter realizzare applicazioni RTEMS.
    \item \textbf{Capitolo 4} - in questo capitolo vengono descritti i drivers di RTEMS e come le ho provate.
    \item \textbf{Capitolo 5} - in questo capitolo viene descritto ciò che si vuole testare, in che modo si vuole testare e i motivi per cui si vuole testare quella specifica interfaccia. Perciò vengono descritti i componenti aggiuntivi forniti da Microchip, e l'architettura generale dei programmi di prova.
    \item \textbf{Capitolo 6} - Ringraziamenti.
    \end{itemize} 
La tesi sarà utile alla BIS-Italia, fazione italiana della British Interplanetary Society società storica britannica,di cui sono membro, che mi ha seguito durante lo stage e integrerà ciò al progetto di creazione di una replica in scala 1:3 di ExoMars Rover che verrà utilizzato per divulgazione.
Tutto il lavoro è stato svolto con l'aiuto dei membri della BIS-Italia e la collaborazione con Microchip. 

\
\tableofcontents
\newpage
\pagenumbering{arabic}

\chapter{Introduzione}
\section{Obiettivi}
\section{Contesto di riferimento}
\section{Applicazione}
\newpage
\chapter{Tecnologie}
\section{RTEMS}
\section{Raspberrypi}
\section{Eclipse}
\newpage
%\chapter{Porting di RTEMS su Rpi}
%\newpage
%\chapter{Integrazione HW e SW}
%\newpage
%\chapter{Attività sperimentale}
%\section{Obiettivi}
%\section{Test set-up}
%\section{Test application SW}
%\section{Risultati finali}
%\newpage
%\chapter{Conclusioni}
%\newpage
%\chapter{Ringraziamenti}

\end{document}