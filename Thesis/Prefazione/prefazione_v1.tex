\documentclass[a4paper,12pt,titlepage,oneside]{book}
\title{Bozza introduzione}
\author{Clark Ezpeleta}
\begin{document}
\maketitle
\newpage
\chapter*{Prefazione}

Lo scopo del mio lavoro consisteva nel porting del sistema operativo RTEMS su Raspberrypi e nella creazione di piccoli programmi per testare il funzionamento dei GPIO, USART, I2C, SPI e l'utilizzo degli interrupts.
Per poter svolgere il porting ho dato una lettura al RTEMS User Manual[1] per comprendere meglio i concetti di RSB(RTEMS Source Builder) e BSP (board support packages), e ho utilizzato la guida fornita da un utente - AlanC,  che ha già svolto il porting per la versione 4.11.
Purtroppo la guida fornita non era totalmente corretta, poichè datata a Marzo 2013 ed era stata scritta per la versione di RTEMS precedente a quella che dovevo utilizzare, cioè la versione 5.1. 
Per questo motivo dopo aver effettuato correttamente il porting ho creato una guida con scritto tutti i passaggi effettuati integrando le correzioni necessarie rilevate. \\
Oltre alla Raspberrypi 3B+ che mi è stata fornita, per poter creare i programmi di test dell'interfaccia I2C e SPI erano necessari dei componenti aggiuntivi che mi sono stati gentilmente offerti dalla Microchip.\\\\
Premesso ciò, ho suddiviso la mia tesi nei seguenti capitoli:

\begin{itemize}
    \item \textbf{Capitolo 1} - in questo capitolo vengono descritti gli obiettivi della tesi, il contesto di riferimento e l'applicazione .
    \item \textbf{Capitolo 2} - in questo capitolo vengono descritte le tecnologie utilizzate.
    \item \textbf{Capitolo 3} - in questo capitolo viene descritta tutta l'attività di porting svolta e la definizione della toolchain per poter realizzare applicazioni RTEMS.
    \item \textbf{Capitolo 4} - in questo capitolo vengono descritti i drivers di RTEMS e come le ho provate.
    \item \textbf{Capitolo 5} - in questo capitolo viene descritto ciò che si vuole testare, in che modo si vuole testare e i motivi per cui si vuole testare quella specifica interfaccia. Perciò vengono descritti i componenti aggiuntivi forniti da Microchip, e l'architettura generale dei programmi di prova.
    \item \textbf{Capitolo 6} - Ringraziamenti.
    \end{itemize} 
La tesi sarà utile alla BIS-Italia, fazione italiana della British Interplanetary Society società storica britannica,di cui sono membro, che mi ha seguito durante lo stage e integrerà ciò al progetto di creazione di una replica in scala 1:3 di ExoMars Rover che verrà utilizzato per divulgazione.
Tutto il lavoro è stato svolto con l'aiuto dei membri della BIS-Italia e la collaborazione con Microchip. 



\end{document}